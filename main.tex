\documentclass{article}
\usepackage[utf8]{inputenc}

\title{Proof of Concept: Workflow process}
\author{Ellen Kirkpatrick }
\date{September 2019}

%This sets the indent length of a paragraph 
\setlength{\parindent}{0em}

%This sets the whitespace between paragraphs
\setlength{\parskip}{1em}

\begin{document}

\maketitle

\section{Proof of Concept}
The aim of this project is to view multiple sources at the same time in one place in order to draw connections between them. Depending on their shared themes, they will be tagged and annotated accordingly and stored using the same program for ease of future access.\\
This is helpful in the preparation stages of research when finding initial sources, and new paths for further study. Following these steps will mean that sources can be referred back to more efficiently in later stages of the research process and there will be appropriate notes to indicate why the source was useful and what information it contains.\\
It will also make creating a bibliography list more easier as metadata from sources will be stored and transferred to publishing programs when it comes to submitting research.

\section{Programs used}
\subsection{Voyant}
This project uses Voyant within the web browser. This can be accessed here: https://voyant-tools.org/. For reliability purposes, Voyant tools was deemed more reliable than Voyant server which is installed and worked offline on the computer hard drive. While this is also an option, it did not perform consistently in tests so the browser was chosen as the better option.
\subsection{Zotero}
Zotero has been selected as a storage program. The program was installed from here: https://www.zotero.org/. It works offline on the hard drive once installed and also connects to the browser, an additional connector must be installed for this. Mozilla Firefox was chosen for this particular case. 

\section{Set up}
The following steps are required in order to ensure efficiency for completing the proof of conept:
\begin{itemize}
    \item Install Zotero from https://www.zotero.org/
    \item Select browser and install the browser connector. This is so Zotero can access files straight from the browser and save them to the library.
    \item Check that the Zotero connector appears in the top right corner above the web browser toolbar. It should appear as a "Z" or an icon image of a piece of paper. Waving the mouse over the icon will show the following message: "Save to Zotero". 
    \item If using Voyant server, the latest version of java must be updated. Can be installed from here: https://www.java.com/en/download/
    \item If using Voyant server, access the download here: http://docs.voyant-tools.org/resources/run-your-own/voyant-server/
    \item If using Voyant tools via a web browser, type the url into the search bar: https://voyant-tools.org/
    \item Add Voyant website to the bookmarks toolbar or save to favourites for easier access in future.
\end{itemize}
Once these steps are complete, the process for identifying shared themes and storing sources is able to begin. 

\section{Workflow Process}
\subsection{Finding and downloading sources}
\begin{enumerate}
\item Open Zotero on computer first. Keep open in background before opening web browser.
    \item Access chosen database or original source online. For the purposes of this project, Macquarie University database was used. (Due to restrictions as a result of privacy and institutional access, there will be some differences).
    \item Search key words for chosen topic, assignment, research idea etc.
    \item Access sources directly through logging in to database server, accessing pdf and clicking on the Zotero connector in the top right corner.
    \item Pdf source will save to Zotero library.
     \item Check Zotero library that source appears. In right hand panel, check that appropriate metadata is there. If source is reliable all metadata will appear (author, data, title, journal, volume & issue number, pages). If metadata is missing, add manually from the source location on database. If metadata is not available, re-consider the reliability of source as it may not be valid for research purposes.
    \item At the same time, download pdf source to a new directory called "test" or something similar on the desktop. Pin this directory to quick access so it can be more efficiently accessed in future. Save all sources to this folder. This directory will be used to upload sources to Voyant. 
  \end{enumerate}

\subsection{Identifying commonalities between sources}
\begin{enumerate}
    \item Open a new browser tab or window, and type in url: https://voyant-tools.org/ to access Voyant tools on browser.
    \item The Voyant homepage should show a box in the middle of the page where text can be copied and pasted. Due to restrictions on institutional access, sources will be uploaded from the computer. However, other sources which have open access can be transferred directly to Voyant through copying and pasting the url.
    \item To upload sources, click on the bottom left button of the textbook called "upload". This will open an upload window.
    \item Click on "test" directory which is pinned to quick access on the left hand side of the upload window. All sources are stored there, select the sources to upload to Voyant for comparison.
    \item Upload sources to Voyant. Select "reveal".
    \item Voyant will redirect to a page which shows all sources. There are 5 separate sections which are all different tools. Voyant will automatically display default tools which are: cirrus, reader, trends, summary and contexts. There is an icon next to each tool where they can be changed to suit the purposes of the student/researcher.
    \item To identify commonalities, the trends tool and reader tool are particularly helpful. These tools show re-occurring key words and themes in the sources and can pinpoint their exact locations. This helps the student/researcher see what topic areas these sources cover. This highlights possible paths for further study or research.
    \item Use the reader tool (if on default settings, will appear in the centre of the top row) to read the abstract or specific part of sources.
\end{enumerate}

\subsection{Annotating sources}
\begin{enumerate}
    \item Using information provided about the sources in Voyant tools, annotate the necessary sources in order to determine relevance, subject area and how it can be used.
    \item Click on Zotero library and locate source. Can use search bar and search via the title or author.
    \item Select source in library. In the right hand panel, click on the button "notes". 0 notes should appear. Select "add".
    \item An internal text editor window will open in the panel, input necessary annotations. The annotations should appear connected to the source with a small icon image of a notepad next to it. The notes save automatically and can be exited by clicking out.
    \item Repeat steps to input annotations for other sources.
    \item To access annotations, click on the source and select "notes" in the right hand panel. It should display any of the inputted annotations and provide options to edit and add more. The annotations will always be displayed connected to the source in the Zotero library.
\end{enumerate}

\subsection{Adding tags}
\begin{enumerate}
    \item Adding tags allows sources to be grouped together by relevance, topic area or common themes. This has been identified through Voyant.
    \item Click on source in Zotero library and in left hand panel select button "tags". Depending on source metadata, Zotero may automatically pick up on tags already attacehd to the source. These can be edited if necessary.
    \item To create a new tag, select "Add". A single text book will appear with a tag icon next to it. Input tag. Tags should be kept short and succinct for maximum efficiency. Multiple tags can be added if the source covers multiple subject areas.
    \item Repeat steps for other sources. 
    \item Using the same tags for other sources will group sources together. If typing the same tag, an option will appear to select pre-existing tags. 
    \item To test whether the tags have been successful, type a tag into the search bar in Zotero. Any sources with that tag should appear.
\end{enumerate}
\end{document}
