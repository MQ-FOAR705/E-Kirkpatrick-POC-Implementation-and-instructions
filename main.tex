\documentclass{article}
\usepackage[utf8]{inputenc}

\title{Proof of Concept: Workflow process}
\author{Ellen Kirkpatrick }
\date{October 2019}

%This sets the indent length of a paragraph 
\setlength{\parindent}{0em}

%This sets the whitespace between paragraphs
\setlength{\parskip}{1em}

\begin{document}

\maketitle

\section{Proof of Concept}
The aim of this project is to view multiple sources at the same time using one program in order to draw connections between them. Depending on their shared themes or other commonalities, they will be tagged and annotated accordingly on a separate reference management program so they can be used more efficiently in future stages of the research process.\\
This is helpful in the preparation stages of research when finding initial sources and new avenues for further study. Following these steps will ensure that sources can be referred back to more efficiently in later stages of the research process and there will be appropriate notes to indicate why the source was useful and what information it contains.\\
It will also make creating a bibliography list more easier as metadata from sources will be stored and transferred to publishing programs when it comes to submitting research.

\section{Programs used}
\subsection{Voyant}
This project uses Voyant tools which can be accessed through web browser using the following URL: $https://voyant-tools.org/.$ Voyant is an open-source, web based tool which can be used for scholarly purposes concerning the reading and analysis of texts.\\
During acceptance criteria tests, Voyant tools was deemed more reliable than Voyant server which can be installed to work offline on a local computer hardrive. While this is also an option, it did not perform consistently in tests due to storage issues so Voyant tools was selected as the more reliable option given the time constraints.\\
However, it should be acknowledged that Voyant server could be used as an alternative to this workflow process and if it works successfully, more parts of the process could be automated.
\subsection{Zotero}
Zotero is the reference and source management program chosen for this workflow process. It is free to the public and can be accessed and installed from the following URL: $https://www.zotero.org/.$\\
Zotero works offline on the hard drive once installed. However, to extract sources and metadata from databases, appropriate internet connection is required. An additional browser connector from the same URL must be installed in order for Zotero to work most efficiently. This connector allows sources and metadata to be extracted directly to the general Zotero library.\\
Zotero can be used on Mac, Windows and Linux and there are browser connectors available for chrome, safari and firefox. This workflow used Mozilla Firefox throughout the entire process. 


\section{Set up}
The following steps need to be completed in order to ensure efficiency for completing the proof of concept:
\begin{itemize}
    \item Install Zotero from $https://www.zotero.org/$
    \item Select browser and install the browser connector. This is so Zotero can access files straight from the browser and save them to the library.
    \item Check that the Zotero connector appears in the top right corner above the web browser toolbar. It should appear as a "Z" or an icon image of a piece of paper. Waving the mouse over the icon will show the following message: "Save to Zotero". 
    \item If using Voyant server, the latest version of java must be updated. Can be installed from here: https:$//www.java.com/en/download/$
    \item If Voyant server has been selected, access the download here: $http://docs.voyant-tools.org/resources/run-your-own/voyant-server/$
    \item If using Voyant tools via a web browser, type the url into the search bar on selected web browser: $https://voyant-tools.org/$
    \item Add Voyant website to the bookmarks toolbar or save to favourites for easier access in future.
\end{itemize}

\textbf{Note:} All of these steps are required for the proof of concept to work.\\
Once these steps are complete, the process for identifying shared themes and storing sources is able to begin. 

\section{Workflow Process}
This section includes a step-by-step guide on how the proof of concept works. It follows the user stories which have been tested against necessary acceptance criteria and quality assurance tests.\\
However, it is important to note here that this workflow process reflects the outcomes of these tests. It may be possible for others to automate parts of this process depending on their own acceptance criteria results.
\subsection{Finding and downloading sources}
\begin{enumerate}
\item Open Zotero on computer first. Keep program open in background before opening web browser.
\item Open selected web browser and search appropriate database.
    \item Access chosen database or original source online. For the purposes of this project, Macquarie University database was used. (Due to restrictions as a result of privacy and institutional access, there will be some differences yet the process will work for other databases if the steps are followed).
    \item Search key words for chosen topic, assignment, research idea etc.
    \item Access sources directly through logging in to database server, accessing pdf and clicking on the Zotero connector in the top right corner.
    \item Pdf source will save to Zotero library.
     \item Check Zotero library that source appears. In right hand panel, check that appropriate metadata is there. If source is reliable all metadata will appear (author, data, title, journal, volume & issue number, pages). If metadata is missing, add manually from the source location on database. If metadata is not available, re-consider the reliability of source as it may not be valid for research purposes.
    \item At the same time, download pdf source to a new directory called "test" or something similar on the desktop. Pin this directory to quick access so it can be more efficiently accessed in future. Save all sources to this folder. This directory will be used to upload sources to Voyant. 
  \end{enumerate}

\subsection{Identifying commonalities between sources}
\begin{enumerate}
    \item Open a new browser tab or window, and type in url: $https://voyant-tools.org/$ to access Voyant tools on browser.
    \item The Voyant homepage should show a box in the middle of the page where text can be copied and pasted. Due to restrictions on institutional access, sources will be uploaded from the computer. However, other sources which have open access can be transferred directly to Voyant through copying and pasting the url.
    \item To upload sources, click on the bottom left button of the textbook called "upload". This will open an upload window.
    \item Click on "test" directory which is pinned to quick access on the left hand side of the upload window. All sources are stored there, select the sources to upload to Voyant for comparison.
    \item Upload sources to Voyant. Select "reveal".
    \item Voyant will redirect to a page which shows all sources. There are 5 separate sections which are all different tools. Voyant will automatically display default tools which are: cirrus, reader, trends, summary and contexts. There is an icon next to each tool where they can be changed to suit the purposes of the student/researcher.
    \item To identify commonalities, the trends tool and reader tool are particularly helpful. These tools show re-occurring key words and themes in the sources and can pinpoint their exact locations. This helps the student/researcher see what topic areas these sources cover. This highlights possible avenues for further study or research.
    \item Use the reader tool (if on default settings, will appear in the centre of the top row) to read the abstract or specific parts of sources.
\end{enumerate}

\subsection{Annotating sources}
\begin{enumerate}
    \item Using information provided about the sources in Voyant tools, annotate the necessary sources stored in Zotero. Annotations should be done accordingly to the relevance and subject area of the source as well as how it can be used in the research (If collaborating on research with others, think about how these annotations can best help them accessing the source in the future so they can understand your point of view). 
    \item Click on Zotero library and locate source. Can use search bar and search via the title or author.
    \item Select source in library. In the right hand panel, click on the button "notes". 0 notes should appear. Select "add".
    \item An internal text editor window will open in the panel, input necessary annotations. The annotations should appear connected to the source with a small icon image of a notepad next to it. The notes save automatically and can be exited by clicking out.
    \item Repeat steps to input annotations for other sources.
    \item To access annotations, click on the source and select "notes" in the right hand panel. It should display any of the inputted annotations and provide options to edit and add more. The annotations will always be displayed connected to the source in the Zotero library.
\end{enumerate}

\subsection{Adding tags}
\begin{enumerate}
    \item Adding tags allows sources to be grouped together by relevance, topic area or common themes. This has been identified through Voyant.
    \item Click on source in Zotero library and in left hand panel select button "tags". Depending on source metadata, Zotero may automatically pick up on tags already attacehd to the source. These can be edited if necessary.
    \item To create a new tag, select "Add". A single text book will appear with a tag icon next to it. Input tag. Tags should be kept short and succinct for maximum efficiency. Multiple tags can be added if the source covers multiple subject areas.
    \item Repeat steps for other sources. 
    \item Using the same tags for other sources will group sources together. If typing the same tag, an option will appear to select pre-existing tags. 
    \item To test whether the tags have been successful, type a tag into the search bar in Zotero. Any sources with that tag should appear.
\end{enumerate}

\section{Dependent variables}
The user stories labelled "finding and downloading sources" and "identifying commonalities between sources" are essential for this workflow process to be successful. They are the core steps as they directly result in sources being analysed according to their content and stored appropriately.\\
The steps "annotating sources" and "adding tags" are not essential for the workflow process to be successful in storing sources effectively in a single program, yet they help with more efficiency and will be useful in future stages of research so it is important to complete them to optimise the results of this workflow process. Both of these steps are reliant on success in downloading sources and finding commonalities.\\
The step "adding tags" is dependent on "annotating sources" as the tags drawn from the notes created on each source.  

\section{Result}
Following this workflow process will result in a Zotero library which is populated with all sources required for research project. As the research process is often long, it is important that appropriate annotations and tags are added to each source when they are initially accessed so that they can be referred to and used more efficiently in future.\\
As the research process continues, this workflow process can be continually repeated in order to build a more extensive library of sources and to compare them more efficiently.\\
Using a storage program also means that all metadata is stored and when it comes to writing up the research or thesis, and submitting it, the metadata can be exported directly to the chosen publishing program through Zotero. This means bibliographic references do not need to be manually typed. Although, good research practice is double checking that all bibliographic references are checked before final submission.

\section{Problems}
My original workflow has been changed due to problems experienced that were not able to be addressed within the time constraints. Firstly, Voyant server was installed and the workflow order began with directing sources from database directly to Voyant and then being saved to Zotero for localised storage. The aim was to use script to automate the process of moving sources directly from Voyant sever to Zotero yet there were difficulties using Voyant server due to space available on the computer.\\
As a result, the workflow process was changed so that sources were directly saved to Zotero from the first step and then re-uploaded to Voyant. \\
Unfortunately, the changes to the workflow resulted in increased time constraints and the process was not automated. This workflow process represents the steps that are required to be taken to fulfil the workflow process to the most efficiency in how it can be in its present form.

\section{Relevant links}
This section provides links to relevant github repositories including documentation relevant to this proof of concept implementation:\\
\\
https://github.com/MQ-FOAR705/E-Kirkpatrick-PoC-project\\
Includes project management, issues, user stories, licence and release.\\
\\
https://github.com/MQ-FOAR705/E-Kirkpatrick-POC-Implementation-and-instructions\\
Includes documented workflow process and instructions (this document).\\
\\
https://github.com/MQ-FOAR705/E-Kirkpatrick---Learning-Journal\\
Includes learning journal documenting steps taken and process.

The latter 2 links can be found as submodules within the parent github repositry E-Kirkpatrick-PoC-project.

\end{document}
