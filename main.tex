\documentclass{article}
\usepackage[utf8]{inputenc}

\title{Proof of Concept: Implementation handbook}
\author{Ellen Kirkpatrick }
\date{September 2019}

%This sets the indent length of a paragraph 
\setlength{\parindent}{0em}

%This sets the whitespace between paragraphs
\setlength{\parskip}{1em}

\begin{document}

\maketitle

\section{Proof of Concept}
The aim of this project is to view multiple sources at the same time in one place in order to draw connections between them. Depending on their shared themes, they will be tagged and annotated accordingly and stored using the same program for ease of future access.

\section{Programs used}
\subsection{Voyant}
This project uses Voyant within the web browser. This can be accessed here: https://voyant-tools.org/. For reliability purposes, Voyant tools was deemed more reliable than Voyant server which is installed and worked offline on the computer hard drive. While this is also an option, it did not perform consistently in tests so the browser was chosen as the better option.
\subsection{Zotero}
Zotero has been selected as a storage program. The program was installed from here: https://www.zotero.org/. It works offline on the hard drive once installed and also connects to the browser, an additional connector must be installed for this. Mozilla Firefox was chosen for this particular case. 

\section{Set up}
The following steps are required in order to ensure efficiency for completing the proof of conept:
\begin{itemize}
    \item Install Zotero from https://www.zotero.org/
    \item Select browser and install the browser connector. This is so Zotero can access files straight from the browser and save them to the library.
    \item Check that the Zotero connector appears in the top right corner above the web browser toolbar. It should appear as a "Z" or an icon image of a piece of paper. Waving the mouse over the icon will show the following message: "Save to Zotero". 
    \item If using Voyant server, the latest version of java must be updated. Can be installed from here: https://www.java.com/en/download/
    \item If using Voyant server, access the download here: http://docs.voyant-tools.org/resources/run-your-own/voyant-server/
    \item If using Voyant tools via a web browser, type the url into the search bar: https://voyant-tools.org/
    \item Add Voyant website to the bookmarks toolbar or save to favourites for easier access in future.
\end{itemize}
Once these steps are complete, the process for identifying shared themes and storing sources is able to begin. 

\section{Process}
\subsection{Finding and downloading sources}
\begin{enumerate}
\item Open Zotero on computer first. Keep open in background before opening web browser.
    \item Access chosen database or original source online. For the purposes of this project, Macquarie University database was used. (Due to restrictions as a result of privacy and institutional access, there will be some differences).
    \item Search key words for chosen topic, assignment, research idea etc.
    \item Access sources directly through logging in to database server, accessing pdf and clicking on the Zotero connector in the top right corner.
    \item Pdf source will save to Zotero library.
    \item Check Zotero library that source appears. In right hand panel, check that appropriate metadata is there. If source is reliable all metadata will appear (author, data, title, journal, volume & issue number, pages). If metadata is missing, add manually from the source location on database. If metadata is not available, re-consider the reliability of source as it may not be valid for research purposes.
\end{enumerate}

\end{document}
